\cvsection{项目经历}
\begin{cventries}
  \cventry
    {华东师范大学 \quad 数据科学与工程实验室  } 
    {分布式内存数据库  \it{CLAIMS(Ginkgo)}}
    {上海}
    {2015.05 - 2018.01}
    {
      \begin{cvitems}
	\item {项目介绍:CLAIMS是分布式内存数据库,OLAP,有支持实时注入分析查询,支持弹性流水线等特点。}
	\item {工作内容:数据库解析层的重构开发;数据库的执行引擎的部分实现;分布式JOIN算法实现及优化。部署了CLAIMS的集群监控系统,基于\it{influxdb+collectd+grafana}。}
	\item {相关技能:分布式系统,数据库,\it{Linux, C++}}
      \end{cvitems}
    }

    \cventry
    {星环信息科技}
    {列式存储系统Holodesk}
    {上海}
    {2018.03 - today}
    {
      \begin{cvitems}
        \item {项目介绍:ArgoDB是一款分布式分析型数据库,Holodesk为其默认存储。}
        \item {团队规模:4人}
      \end{cvitems}
    }

    \cventry
    {\textbf{开源spark读取Holodesk表生成DataFrame}}{}{}{}
    {
        \begin{cvitems}
          \item {\textbf{工作内容:}该项目主要是为了解决客户对其基于开源spark的应用想读取holodesk表的数据用于计算,不使用ArgoDB的计算引擎。采用了两种方式来用开源spark读取Holodesk表文件,第一个用开源spark的DataSource API,这种方式相当于只提供数据源,计算逻辑需要另外提供。第二种是通过自定义优化规则,修改spark的执行计划,同时实现对应的ScanRDD。本人主要负责第一种方式以及第二种方式中ScanRDD的实现,另一位同事负责注册规则部分。}
        \end{cvitems}
    }

    \cventry
    {\textbf{基于SuRF的索引}}{}{}{}
    {
        \begin{cvitems}
          \item {\textbf{工作内容:}该项目主要是提供新的轻量级索引,来加速根据范围条件,检索文件的速度。SuRF来自2018SIGMOD会议上的best paper, 是一种高效的range filter。独立完成了一个简单版本的SuRF,即采用LOUDS-Sparse编码的SuRF-base。主要用于快速判断文件中是否包含给定范围或值。}
	  \item {这里测试发现,SuRF所占用的存储空间仍比较大,在点查方面布隆过滤更加合适,而且范围查询受到前缀集合的特性影响,并不是十分通用。}
        \end{cvitems}
    }

    \cventry
    {\textbf{流数据入分析型数据库}}{}{}{}
    {
        \begin{cvitems}
          \item {\textbf{工作内容:}有很多客户的业务场景是对流数据实时处理和计算的基础上,还需要对这些数据或者其计算后的结果集进行复杂的查询和分析。ORC和HBASE做复杂的统计分析比较耗时,因此考虑直接存入ArgoDB中进行分析。在spark中实现了相应的Actor,基于两阶段的提交的模型,将一定时间窗口内的数据持久化到存储层。}
        \end{cvitems}
    }

\end{cventries}

\cvsection{其他项目}
\begin{cventries}
  \cventry
    {Infosys Instep Intern}
    {印孚瑟斯}
    {印度,班加罗尔}
    {2016.07 - 2016.09}
    {
      \begin{cvitems}
        \item {参与实现一个分布式实时注入查询的内存数据库, 负责执行引擎的设计和实现。完成了基于迭代器模式的SQL执行引擎原型,主要模块有语句执行器,语法解析器,逻查询计划,物理查询计划等。}
      \end{cvitems}
    }
\end{cventries}


